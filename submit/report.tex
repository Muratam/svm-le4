\documentclass[]{jsarticle}
\usepackage[dvipdfmx]{graphicx}
\usepackage[deluxe]{otf}
\renewcommand{\kanjifamilydefault}{\mgdefault}
\newcommand*{\graphc}[3][7.0cm]{\begin{figure}[h] \begin{center}\includegraphics[clip,width= #1]{#2}\caption{#3}\end{center}\end{figure}}
% http://www.ai.soc.i.kyoto-u.ac.jp/~matsubara/le4-2016/index.php?%28Step1%29%20SVM%E3%81%AE%E4%BD%9C%E6%88%90


\begin{document}
\title{平成28年度 3回生後期実験(エージェント) \\ 課題1 SVMの作成 }
\author{村田 叡}
\date{ 2016/10/14 }
\maketitle

\section{プログラム概要}
この課題ではサポートベクターマシン(SVM)のpython3による実装を行った。
1 と -1 の2クラスのサンプル点集合を分類する評価機を生成する。
線形評価関数又はカーネルトリックを用いた評価関数を生成する。
外部ライブラリとして凡庸行列計算に numpy,プロッティングに matplotlib,
二次計画問題の計算に cvxopt を用いた。
以下にその詳細を述べる。



\section{外部仕様}
\subsection{svm.py}
今回のコードは svm.py にて実装した。
このコードは、引数としてサンプル点集合のファイル名、カーネル名、及びオプションをとる。
サンプル点集合の形式については、実験のページに書かれてあるものに従った。
カーネル名は、gauss, polynomial, sigmoid, linear のいずれかをとる。
デフォルトのカーネル名は gauss である。
--plot オプションを使うと結果をmatplotlibでプロットしたものを表示する。
\subsection{実行例と実行結果}
\begin{verbatim}
# 必要なライブラリの導入
$pip3 install -r requirements.txt
# サンプル点集合 sample_circle.dat のガウスカーネルのSVMを作成し、プロッティングする。
$python3 svm.py sample_data/sample_circle.dat gauss --plot
>> α : [  0.00000000e+00   0.00000000e+00   0.00000000e+00   0.00000000e+00 ...
>> θ : 3.56258579268
>> passed :100 / 100
>> f(x) =  +14.8961965139*K([31.0, 12.0],x) -6.46323001951*K([40.0, 24.0],x) ...
\end{verbatim}
上記のように、各αの値、θの値、サンプル点による識別器の識別率、識別器の関数が結果として得られる。
なお、f(x) の K は 各カーネルを指す。
\graphc[16cm]{./images/circleplot.png}{ガウスカーネルのSVM}


\section{内部仕様}
以下ではsvm.pyでの大域変数と各関数について説明する。


\section{考察}
% 機能の実装が目的であれば,それが実現できていることを示します.
% 性能評価であれば,単に一例を示してうまく動作したなどとまとめるのではなく,
% データ数などの条件を変えるなど,多面的に評価します.

\section{感想}
% 評価結果からどのような傾向が読み取れるかを記述.どういう問題点があって,どう解決したかなど.

\end{document}

%提出物:ソースコード,プログラム仕様,評価結果
%ソースコードは,ライブラリ,識別器作成に用いる訓練データセット,Makefile等の設定などを含めること.
%Makefileを作ってコンパイルをできるようにすること.
%ソースコード,プログラム仕様,評価結果を1つのディレクトリにまとめてzipあるいは.tar.gz形式にすること.
%提出先:松原(matsubara at i.kyoto-u.ac.jp),山本(s-yamamoto at ai.soc.i.kyoto-u.ac.jp)
%課題の提出は一つのメールで教員とTAに同報してください.別メールで送られると,教員TA間での無駄な確認作業が増えます.
%メールのSubjectは,ex4 課題番号 名前(例えば,ex4-1-matsubara)とすること.
%メール本文で自身の名前を名乗ること.本文が何もないと迷惑メールに分類されることがあります.
%△△先生,TA▽▽さん,□□さん,○○です.         
%課題1を提出します.よろしくお願いします.