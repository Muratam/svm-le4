\documentclass[]{jsarticle}
\usepackage[dvipdfmx]{graphicx}
\usepackage[deluxe]{otf}
\usepackage[hang,small,bf]{caption}
\usepackage[subrefformat=parens]{subcaption}
\captionsetup{compatibility=false}

\renewcommand{\kanjifamilydefault}{\mgdefault}
\newcommand*{\graphs}[3]{
  \begin{minipage}[b]{0.5\hsize}
    \includegraphics[scale=#1]{#2}
    \subcaption{#3}
  \end{minipage}
}

\begin{document}
\title{平成28年度 3回生後期実験(エージェント) \\ 課題4 入札エージェントの作成 }
\author{村田 叡}
\date{ 2016/11/18 }
\maketitle

\section{プログラム概要}
今回のレポートではオークションで予測値を元に入札するエージェントを作成した。

\subsection{プログラムの起動方法及び実行例}
readme.md  の項 requirements を参照のこと

\section{外部仕様}
エージェントはPython3で実装した。
以下、そのPython3の外部インターフェイスについて述べる

\subsection{py3/multiagent.py}
このコードでは、多エージェントによるオークションシミュレートを実装している。
オークションデータのcsvファイル、エージェントの種類を引数にとり、オークションを行う。 \\
エージェントは以下の4種類を用意している。 \\
1. 一日目の購入価格の中央値で購入し続ける単純エージェント @simple \\
2. 一日目の購入価格の中央値の5倍の価格で購入し続ける貪欲エージェント @greedy \\
3. 一つ前の購入価格を次の価格の予測値として利用して戦略的に購入するエージェント @sorena \\
4. SVRに基づいて次の価格を予測し、戦略的に購入するエージェント @svr \\
例えば、id0001.csvを用いて、 貪欲エージェントとSVRエージェントを戦わせるには \\
\verb|python3 py3/multi_agent.py sample_data/id0001.csv @greedy @svr|
のようにすればよい。 \\
\verb|--show-process| を引数に加えると、購入過程を逐一表示する。
この購入過程は、例えば \verb|0 145.16 13.138| については、
左から 購入者id(0), 購入された時間(145.16)、落札価格(13.138) を表す。
購入者idは、csvのデータの人が買えば -1を,
0以上であれば、それぞれエージェントを順に表す。
例えば、\verb|python3 py3/multi_agent.py sample_data/id0001.csv @greedy @svr @simple --show-process|
 によって実行すれば、@greedyは0番、 @svrは1番、@simpleは2版である。
SVRの予測値については、ガウスカーネルで、一日目のデータを元に求めた C,σの値を元にして、
最近のデータから毎度SVRを作成して予測している(教師データは100個)。

\section{内部仕様}
\subsection{py3/visualize.py}
このコードでは、データをプロットして可視化するために必要な関数を実装している。
CUIインターフェースは無い。
3Dデータを表示する関数、SVRの予測がどのようなものなのかを検証する関数を提供する。

\subsection{py3/auction.py}
このコードでは、オークションデータを取り扱うAuctionクラスを実装している。
CUIインターフェースは無い。
Auctionクラスは、コンストラクタにCSVファイルをとってデータを読み込み、
可視化する関数や、価格の配列を取り出す関数などを提供している。


\subsection{py3/multiagent.py}
このコードでは、内部的には購入に携わるBuyerクラス及び購入戦略に基づいて購入するAgentクラスを実装している。
\subsubsection{Buyerクラス}
最初に所持金額を与え、buyメソッドにより購入を行うことができるような抽象化をしている。
\subsubsection{Agentクラス}
一日目のデータ、一日目の購入価格の中央値、Buyerクラスインスタンスを元にして作成する。
buy\_から始まる関数では、各エージェントがその関数が表す戦略に基づいた購入価格を出力する。
do\_multi\_auction 関数によって、実際に多エージェントによるオークションをシュミレートする。

\section{考察}
\subsection{SVRの予測精度について}
SVR自体の能力を過信せず、一度検証してみることが大事である。
予測精度を可視化するために、ある関数$f(x)$を定義し($x \in [0,1]$ )、
与えるデータの範囲を$[0,b](0<b<1)$とすることで予測がどのように
変化するかを考察する。
なお、この考察において教師データ数は b * 200 である。
\newpage
\subsubsection{sin波}
\begin{figure}[htbp]
 \graphs{0.4}{./plots/sin1.png}{$b = 0.05$}
 \graphs{0.4}{./plots/sin2.png}{$b = 0.5$}
 \caption{$f(x)=sin(10x)$ による予測値(赤:実値,青:予測値) }
 \graphs{0.4}{./plots/2sin1.png}{$b = 0.05$}
 \graphs{0.4}{./plots/2sin2.png}{$b = 0.5$}
 \caption{$f(x)=sin(30x)$ による予測値(赤:実値,青:予測値) }
\end{figure}
sin波は、規則性が強く、とても少ないサンプル数でもほとんど正しい予測ができることがわかる。

\newpage
\subsubsection{ その他の100件($b=0.5$)以下の教師データによる関数の予測 }
\begin{figure}[htbp]
 \graphs{0.4}{./plots/func1.png}{$round(20x) + 2sin(20x) ,(b = 0.2)$}
 \graphs{0.4}{./plots/func2.png}{$exp(x),(b = 0.3)$}
 \graphs{0.4}{./plots/func3.png}{$x+round(x),(b = 0.3)$}
 \graphs{0.4}{./plots/func4.png}{$x+0.2 *(round(x)+random),(b = 0.3)$}
 \caption{赤:実値,青:予測値 (randomは、0~1の一様乱数を、roundは四捨五入を表す関数)}
\end{figure}
ある程度の規則性があれば、ほとんど正しい予測ができていることが分かる。
特に図(c)のように、急激に変化した場合も、
ある程度は急激に変化した値を外れ値とみなして急な追随をしていないところが、
SVRの予測精度の高さをよく示している。
図(d)のように、ノイズの乗ったものに関しても、ノイズの範囲で正しく予測できている事がわかる。

\newpage
\subsection{オークションについて}
以下、課題にあるオークションについて考察する。
\subsection{オークションのデータについて}
まず、時系列データの妥当性について考える。
例えば、毎日のデータについて朝7時は安い、などのように、時刻に関連して値段が変わりやすい
場合、日にちに関する情報を与えたほうがよくなる。しかし、以下の図のように、
同時刻でも値段は全く異なり、日にちには依存がないように見えるため、
時系列データとして扱う方法は妥当であるといえる。
\begin{figure}[htbp]\begin{center}
 \graphs{0.4}{./plots/prices.png}{奥行きが日程を表す}
\end{center}\end{figure}

\subsection{エージェントについて}
エージェントは前述の通り以下の4種類を用意している。 \\
1. 一日目の購入価格の中央値で購入し続ける単純エージェント @simple \\
2. 一日目の購入価格の中央値の5倍の価格で購入し続ける貪欲エージェント @greedy \\
3. 一つ前の購入価格を次の価格の予測値として利用して戦略的に購入するエージェント @sorena \\
4. SVRに基づいて次の価格を予測し、戦略的に購入するエージェント @svr \\
特に、@sorena は @svr と比べて、予測精度の高さによって購入数がどう変化するかを
表すのによく適している。

\newpage
\subsection{課題4-1,落札額データに従う入札者との勝負}
まず、予測値を元に行う戦略について考える。
予測精度が高ければ高いほど購入が最適になる戦略が望ましい。
購入の期待値 $exp$ を 残金 ÷ 本日の残り出品回数 と定義する。
$exp$ どおりに買うことができれば、最大限大きな金額を提示しながら
ぴったりオークションの回数で購入することができる。
更に、この値を参考にすることで、残りオークション回数が少ないのに残金が余っていれば、
多少高くても買うことができる。
実際にはこの通りに購入しようとしても、全て購入することは難しいので、以下の戦略を取る。
すなわち、予測値が $exp$ 以下だと、 $exp$ で購入することを宣言し、
予測値が $exp * 3$以下なら 予測値の1.1倍、
$ exp * 3 $ 以上だと、それ以上はお金を出せないので $exp*3$を宣言する という
ものである。
なお、この3はヒューリスティックであるが、この値は数倍程度でははあまり性能に寄与しないことを追記しておく。

予測値の1.1倍を使う部分について、
予測値を使わずに $exp$の定数倍を宣言する方法もあるが、
その方法の場合、 安く買える時に安く購入することができなくなる割合が増えるため、
結果的に損をしてしまう。よって予測値を使うことにしている。
また、予測値どおりだと、購入する確率は半分になりお金を余らせがちなので、1.1倍している。

\newpage
\subsubsection{購入結果}
初期所持金額は10000としている。
課題では5種類の商品時系列データが与えられているが、
id0003,id0004に関してはオークション回数が少ないため省略する。
\begin{table}[htb]\begin{center}
  \caption{id0001 理論上最大で 1457 中 628 個買うことができる \\
            (括弧内は参考として残金を最大価格22.93で使い切った場合の購入数を示す)}
  \begin{tabular}{|c|c|c|c|c|} \hline
    エージェント名 & 購入数 & 残金 & 最大購入値 & 最小購入値 \\ \hline \hline
    @simple & 308(607) & 6860.49 & 12.48 & 0.05 \\ \hline
    @greedy & 543 & 1.32 & 22.93 & 0.05 \\ \hline
    @sorena & 482(548) & 1519.09 & 21.97 & 0.05 \\ \hline
    @svr & 620(627) & 173.14 & 21.73 & 0.05 \\ \hline
  \end{tabular}
\end{center} \end{table}

\begin{table}[htb]\begin{center}
  \caption{id0002 理論上最大で 1709 中 820 個買うことができる \\
            (括弧内は参考として残金を最大価格25.0で使い切った場合の購入数を示す)}
  \begin{tabular}{|c|c|c|c|c|} \hline
    エージェント名 & 購入数 & 残金 & 最大購入値 & 最小購入値 \\ \hline \hline
    @simple & 541(696) & 3886.92 & 11.39 & 0.05 \\ \hline
    @greedy & 543 & 1.32 & 22.93 & 0.05 \\ \hline
    @sorena & 722(776) & 1373.71 & 13.9 & 11.26 \\ \hline
    @svr & 772 & 6.20 & 14.06 & 11.26 \\ \hline
  \end{tabular}
\end{center} \end{table}

\begin{table}[htb]\begin{center}
  \caption{id0005 理論上最大で 2193 中 554 個買うことができる \\
            (括弧内は参考として残金を最大価格25.0で使い切った場合の購入数を示す)}
  \begin{tabular}{|c|c|c|c|c|} \hline
    エージェント名 & 購入数 & 残金 & 最大購入値 & 最小購入値 \\ \hline \hline
    @simple & 539 & 2.12 & 22.06 & 0.05 \\ \hline
    @greedy & 436 & 0.58 & 25.0 & 0.05 \\ \hline
    @sorena & 506(522) & 418.33 & 20.37 & 0.05 \\ \hline
    @svr & 545 & 0.80 & 20.54 & 0.05 \\ \hline
  \end{tabular}
\end{center} \end{table}

id0002では、@svrは()内の値では@sorenaに負けている。
理由としては、突然現れる最小購入値を見逃してしまったことが考えられる。
そのほかのものに関しては@svrが一番購入できている。
@simpleはしきい値以下のものしか購入しないため、
しきい値以下のものが全然出ないパターンに関しては非常に弱くなる。
@greedyはお金が尽きるまで購入し続けるため、
はじめに高いものが重なるパターンに関しては非常に弱くなる。
@sorenaはある程度の予測はできるが、常に一歩遅れた値を予測するため、
買い逃す機会が増えてしまう。

\newpage
\subsubsection{購入過程}
\begin{figure}[htbp]
 \graphs{0.4}{./plots/sorena1.png}{@sorenaのid0001購入過程}
 \graphs{0.4}{./plots/svr1.png}{@svrのid0001購入過程}
 \graphs{0.4}{./plots/sorena5.png}{@sorenaのid0005購入過程}
 \graphs{0.4}{./plots/svr5.png}{@svrのid0005購入過程}
 \caption{ 0~25の軸は購入金額を、0.0~-1.0の軸は、手前0.0の列がエージェントを、奥-1.0の列がcsvデータによる購入者を表し、140~280の軸は、購入時刻を表す。}
\end{figure}
@simple,@greedyに関しては、しきい値以下のもののみ買う、
はじめから全て買うなどわかりきっているので省略する。
@sorena の購入過程については、
初めに安く設定しすぎてデータによる購入者がとても安い価格で落札してしまい、
更にその落札額を予測値として利用するために、はじめはほとんど買えないことが分かる。
一方、SVRでは、前半でも安いものをちょくちょくいい感じに購入し、
後半は余らさないように購入頻度を増やしていることが見て取れる。

\newpage
\subsection{課題4-2,多人数の勝負}
他人の広告掲示価値は非常にわかりにくいため、
4-1のモデルをそのまま適応したものについて考察する。
すなわち、データに基づく購入者のほかに、 @simple,@sorena,@svr の3名がいる場合の
動向について考察する。
\begin{figure}[htbp]
 \graphs{0.4}{./plots/group1.png}{id0001購入過程}
 \graphs{0.4}{./plots/group2.png}{id0002購入過程}
 \caption{奥から順に データによる購入者,@simple,@sorena,@svr を表す}
\end{figure}
@sorena と @svr を組み合わせた場合、両者が依存しあい、結果的にほぼおなじ購入曲線を描くことが分かる。
また、 @simple では、購入金額が一日目と大きく変わり、ほとんど購入できていないことが分かる。



\end{document}